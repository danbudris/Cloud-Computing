\documentclass[runningheads,a4paper]{llncs}

\usepackage{amssymb}
\usepackage{url}
\usepackage{times}
\usepackage{float}
\usepackage[T1]{fontenc}

\usepackage{graphicx}
\usepackage{color}
\usepackage{soul}  
\usepackage{nameref}  
\usepackage{amsbsy}  
\usepackage{bezier}  
\usepackage{colortbl}  
\usepackage[leqno,fleqn]{amsmath}  
\usepackage{verbatim}
\usepackage{listings}

\usepackage[utf8]{inputenc}


\setcounter{tocdepth}{3}
\newcommand{\keywords}[1]{\par\addvspace\baselineskip
\noindent\keywordname\enspace\ignorespaces#1}

\begin{document}
\mainmatter

\title{Review of Paper: Main Title of the paper. }
\subtitle{Paper Review }
\date{Spring 2018}

\author{John Doe and Max Master}

\institute{
Boston University\\ Metropolitan College \\ Computer Science Department \\
 \{John.Doe,Max.Master\}@bu.edu \\
% \url{http://www.bu.edu }
}

\maketitle
\begin{abstract}
Hier bitte ein kurzes Abstrakt schreiben.

\end{abstract}

\keywords{Logische Programmierung, Logik, Seminararbeit}

\section{Introduction}

Write a brief introduction about the paper 

You can cite papers like this \cite{Genesereth:1985:LP:4284.4287} and  \cite{Sakama:2008:CAS:1342991.1342993}.


\section{Motivation}




\section{Section title}
 $\{q_1,...q_n\}$. 
 
 $Q_a \leftarrow  q_1 \wedge ... \wedge q_n$, 
 
 $Q_a \leftarrow  q_1 \vee ... \vee q_n$, 

\section{Here is a section}

\subsection{Another Subsection }




\section{Conclusion}
Here write your conclusion 


\bibliographystyle{plain}
\bibliography{literature}
\end{document}
